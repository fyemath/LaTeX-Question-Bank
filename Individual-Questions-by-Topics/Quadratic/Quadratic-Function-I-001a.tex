% !TEX root = ../MA119-Question-Bank.tex





\pgfmathtruncatemacro{\hh}{-random(5,1)}
\pgfmathtruncatemacro{\kk}{-random(1,5)}
\pgfmathtruncatemacro{\xx}{0}
\pgfmathtruncatemacro{\yy}{random(1,5)}
\pgfmathtruncatemacro{\anum}{\yy-\kk}
\pgfmathtruncatemacro{\aden}{(\xx-\hh)^2}




A quadratic function has its vertex at the point $(\hh, \kk)$.  The function
passes through the point  $(\xx,\yy)$.   When written in vertex form, the
function is $f(x)= a(x-h)^2 +k$. Find the $a$, $h$ and $k$.

\begin{solution}
Since the vertex is $(\hh, \kk)$, we know that $h=\hh$ and $k=\kk$. To find $a$, we plug the point $(\xx, \yy)$ in the equation and solve for $a$.
\[
\begin{split}
\yy    &=a(\xx-\display{\hh})^2+\display{\kk}\\
\yy-\display{\kk}&=a(\xx-\display{\hh})^2\\
\anum   &=\aden a\\
a      &=\rdfrac{\anum}{\aden}.
\end{split}
\]
\end{solution}
