% !TEX root = ../MA119-Question-Bank.tex



\pgfmathtruncatemacro{\a}{random(1,5)} 
\pgfmathtruncatemacro{\b}{random(1,3)} 

\pgfmathtruncatemacro{\x}{random(3,8)-6} 

\pgfmathtruncatemacro{\n}{-random(2,5)} 



\pgfmathtruncatemacro{\y}{\n*(\x-\a)+\b}


\pgfmathtruncatemacro{\mnum}{\y-\b}

\pgfmathtruncatemacro{\mden}{\x-\a} 

\pgfmathtruncatemacro{\bnum}{\b*\mden-\mnum*\a}
\pgfmathtruncatemacro{\bden}{\mden} 

\pgfmathtruncatemacro{\bsign}{\bnum*\bden}
 


Find an equation of the line passing through the following two points
\[(\a, \b)\text{\quad and\quad} (\x, \y).\]
Write the equation in the slope-intercept form if the slope is not undefined. 


\begin{solution}
The slope is 
\[
m=
\ifnum\y<0 
	\frac{(\y)-\b}{\x-\a} 
\else
	\frac{\y-\b}{\x-\a}
\fi
=\simplyfrac{\mnum}{\mden}.
\]

\ifnum\mnum=0
	Since the slope is $0$, the line is a horizontal line and the slope-intercept form equation is 
	\[y=\b.\]
\else
	\ifnum\mden=0
		Since the slope is undefined, the line is a vertical line and an equation is 
		\[x=\a.\]
	\else
		We can then write down the point-slope form equation of the line.
		\[y-\b=\rdfrac{\mnum}{\mden}(x-\a).\]
		Solve for $y$ and simplify the equations, we will get the slope intercept form.
		\[
		\begin{split}
		y-\b&=\rdfrac{\mnum}{\mden}(x-\a)\\
		y & =\rdfrac{\mnum}{\mden}(x-\a)+\b\\
		y&= \rdfrac{\mnum}{\mden}x
		\ifnum\bsign=0 
			.
		\else
			\ifnum\bsign>0
				+\rdfrac{\bnum}{\bden}.
			\else
				\rdfrac{\bnum}{\bden}.
			\fi
		\fi
		\end{split}
		\]
	\fi
\fi		

\end{solution}