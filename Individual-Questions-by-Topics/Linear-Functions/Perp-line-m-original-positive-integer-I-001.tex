% !TEX root = ../MA119-Question-Bank.tex



\pgfmathtruncatemacro{\a}{random(2,7)} 

\pgfmathtruncatemacro{\n}{random(1,5)}


\pgfmathtruncatemacro{\absn}{abs(\n)}


\pgfmathtruncatemacro{\b}{\a*\n} 

% \pgfmathtruncatemacro{\absb}{abs(\a*\n)}


\pgfmathtruncatemacro{\c}{random(1,7)-4} 

\pgfmathtruncatemacro{\x}{random(1,8)-4} 
\pgfmathtruncatemacro{\y}{random(1,8)-4}


% \pgfmathtruncatemacro{\mnum}{\y-\b}
% \pgfmathtruncatemacro{\mden}{\x-\a} 

% \pgfmathsetmacro{\n}{\mnum/\mden}

\pgfmathtruncatemacro{\bnum}{\absn*\y+\x}
\pgfmathtruncatemacro{\bden}{\absn} 

\pgfmathtruncatemacro{\bsign}{\bnum*\bden}
 




Find the slope-intercept form of the equation of the line perpendicular to the line $\b x-\a y =\c$ and passing through $(\x, \y)$.


\begin{solution}

The slope of original line is  
\[
m=\n.
\]

The slope of the perpendicular line is 
\[
m_{\perp}=-\frac{1}{m}=-\frac{1}{\n}.
\]


We can then write down the point-slope form equation of the line.
		\[
			\ifnum\y<0 
				y-(\y) 
			\else
				y-\y
			\fi
			=
			\ifnum\absn =1
				\ifnum\x<0 
					x-(\x) 
				\else
					x-\x
				\fi
			\else
				-\frac{1}{\absn}
					\ifnum\x<0 
						(x-(\x)) 
					\else
						(x-\x)
					\fi
			\fi
		\]
Solve for $y$ and simplify the equations, we will get the slope intercept form.
		\[
		\begin{split}
		\ifnum\y<0 
				y-(\y) 
			\else
				y-\y
		\fi
			&
		=\ifnum\absn =1
				\ifnum\x<0 
					x-(\x) 
				\else
					x-\x
				\fi
			\else
				-\frac{1}{\absn}
					\ifnum\x<0 
							(x-(\x)) 
						\else
							(x-\x)
					\fi
			\fi	
		\\
		y   & 
		=\ifnum\absn =1
				\ifnum\x<0 
					x-(\x) 
				\else
					x-\x
				\fi
			\else
				-\frac{1}{\absn}
					\ifnum\x<0 
						(x-(\x)) 
					\else
						(x-\x)
					\fi
			\fi
		+
		\ifnum\y<0
			(\y)
		\else
			\y
		\fi	
		\\
		y   &
		= \ifnum\absn =1
				x
			\else
				-\frac{1}{\absn}x
			\fi
		\ifnum\bsign=0 
			.
		\else
			\ifnum\bsign>0
				+\rdfrac{\bnum}{\bden}.
			\else
				\rdfrac{\bnum}{\bden}.
			\fi
		\fi
		\end{split}
		\]
\end{solution}