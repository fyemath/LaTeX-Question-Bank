% !TEX root = ../MA119-Question-Bank.tex



\pgfmathtruncatemacro{\a}{random(2,5)} 

\pgfmathtruncatemacro{\b}{random(6,9)}


\pgfmathtruncatemacro{\m}{\a/\b}


% \pgfmathtruncatemacro{\b}{\a*\n} 

% \pgfmathtruncatemacro{\absb}{abs(\a*\n)}





\pgfmathtruncatemacro{\c}{random(1,7)-4} 
\pgfmathtruncatemacro{\x}{random(1,8)-4} 
\pgfmathtruncatemacro{\y}{random(1,8)-4}


% \pgfmathtruncatemacro{\mnum}{\y-\b}
% \pgfmathtruncatemacro{\mden}{\x-\a} 

% \pgfmathsetmacro{\n}{\mnum/\mden}

\pgfmathtruncatemacro{\bnum}{-\b*\x+\a*\y}
\pgfmathtruncatemacro{\bden}{\a} 



\pgfmathtruncatemacro{\bsign}{\bnum*\bden}
 




Find the slope-intercept form of the equation of the line parallel to the line $\a y-\b x=\c$ and passing through $(\x, \y)$.


\begin{solution}

The slope of original line is  
\[
m=\simplyfrac{\b}{\a}.
\]

The slope of the parallel line is 
\[
m_{\parallel}=m=\rdfrac{\b}{\a}.
\]


We can then write down the point-slope form equation of the line.
\[
			\ifnum\y<0 
				y-(\y) 
			\else
				y-\y
			\fi
			=
			\ifdim\m pt =1 pt
				\ifnum\x<0 
					x-(\x) 
				\else
					x-\x
				\fi
			\else
				\rdfrac{\b}{\a}
					\ifnum\x<0 
						(x-(\x)) 
					\else
						(x-\x)
					\fi
			\fi
\]
Solve for $y$ and simplify the equations, we will get the slope intercept form.
\[
\begin{split}
		\ifnum\y<0 
				y-(\y) 
			\else
				y-\y
		\fi
			&
		=\ifdim\m pt =1 pt
				\ifnum\x<0 
					x-(\x) 
				\else
					x-\x
				\fi
			\else
				\rdfrac{\b}{\a}
					\ifnum\x<0 
							(x-(\x)) 
						\else
							(x-\x)
					\fi
			\fi	
		\\
		y   & 
		=\ifdim\m pt =1 pt
				\ifnum\x<0 
					x-(\x) 
				\else
					x-\x
				\fi
			\else
				\rdfrac{\b}{\a}
					\ifnum\x<0 
						(x-(\x)) 
					\else
						(x-\x)
					\fi
			\fi
		+
		\ifnum\y<0
			(\y)
		\else
			\y
		\fi	
		\\
		y   &
		= \ifdim\m pt =1 pt
				-x
			\else
				\rdfrac{\b}{\a}x
			\fi
		\ifnum\bsign=0 
			.
		\else
			\ifnum\bsign>0
				+\rdfrac{\bnum}{\bden}.
			\else
				\rdfrac{\bnum}{\bden}.
			\fi
		\fi
\end{split}
\]
\end{solution}