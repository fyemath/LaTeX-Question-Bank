% !TEX root = ../MA119-Question-Bank.tex


%%%%%%%%%%%%%%%% Setup %%%%%%%%%

\pgfmathtruncatemacro{\aa}{random(9,1)} 
\pgfmathtruncatemacro{\bb}{random(1,9)} 
\pgfmathtruncatemacro{\ab}{\bb-\aa}

\ifnum\ab=0 
	\pgfmathtruncatemacro{\cc}{\bb+1} 
\else 
	\pgfmathtruncatemacro{\cc}{\bb} 
\fi
%%%%%%%%%%%%%%%%%%%%%%%%%%%%%%%%%%%%%%%%


%%%%%%%%% For solution %%%%%%%%%%%%%%%%%%%%
\ifnum\ab<0  %%% a>b=c.
	\pgfmathtruncatemacro{\ans}{\aa} 
\else %%% a<=b<=c
	\pgfmathtruncatemacro{\ans}{\cc} 
\fi

%%%%%%%%%%%%%%%%%%%%%%%%%%%%%%%%%%%%%%%%



Solve the following inequality. Write the answer in interval notation.
\[ x > \aa  \mbox{ and }  x > \cc \]

\begin{solution}
	For ``and'', we look for the ``overlap'', that is double shaded portion on the number line. Note that \ifnum\ab<0 $\aa>\cc$ \else $\aa<\cc$ \fi. From the following graph, 
	\begin{center}
	\begin{tikzpicture}
	\draw[-latex] (-3,0)--(3, 0);
	\draw[(-, thick] (-1.5, 0)--(3,0);
	\draw[(-, thick] (0, 0)--(3,0);
	\fill[pattern=north east lines] (-1.5,0) rectangle (3,0.1);
	\fill[pattern=north west lines] (0,0) rectangle (3,0.1);
	\ifnum\ab<0 
	    \node[below] at (-1.5, -0.1) {$\cc$};
	    \node[below] at (0, -0.1) {$\aa$};
    \else 
	    \node[below] at (-1.5, -0.1) {$\aa$};
	    \node[below] at (0, -0.1) {$\cc$};
    \fi
	\end{tikzpicture}
\end{center}
we know that the interval is $(\ans, \infty)$.
\end{solution}
