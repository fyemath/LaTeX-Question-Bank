% !TEX root = ../MA119-Question-Bank.tex



% \pgfmathdeclarerandomlist{varx}{{x}{y}{p}{q}{z}}
\pgfmathdeclarerandomlist{varx}{{x}}
\pgfmathrandomitem{\varx}{varx}
\edef\varx{\varx}

\pgfmathdeclarerandomlist{vary}{{y}}
\pgfmathrandomitem{\vary}{vary}
\edef\vary{\vary}



\pgfmathsetmacro{\b}{int(random(4,5))}
\pgfmathsetmacro{\c}{int(random(6,8))} %%%%%c\neq b 
\pgfmathsetmacro{\d}{int(random(1, 3))}

\pgfmathtruncatemacro{\a}{random(3, 7)}



\pgfmathtruncatemacro{\ac}{(\a)-(\c)}

\pgfmathtruncatemacro{\adbc}{-(\a)*(\d)+(\b)*(\c)}



%%%(x^2+a)/(x+b)(x+c) - (x+d)/(x+c)=-(b+d)/(x+b) 

Solve the  rational equation
\[
   \frac{\a}{x-\b} - \frac{\c}{x-\d} = 0 .
\]

\begin{solution}
\[
	\begin{split}
		\frac{\a}{x-\b} - \frac{\c}{x-1} &= 0\\
    (x-\d)(x-\b)\frac{\a}{x-\b} - (x-\d)(x-\b)\frac{\c}{x-\d} & =0\\
    \a(x-\d)-\c(x-\b) &=0\\
    \invisibleoneincoef{\ac}x+\adbc&=0\\
    x&= \reducedfraction{-\adbc}{\ac}
	\end{split}
\]

\end{solution}