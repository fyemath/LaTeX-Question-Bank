% !TEX root = ../MA119-Question-Bank.tex



\pgfmathtruncatemacro{\a}{random(1,3)} 
\pgfmathtruncatemacro{\bb}{random(1,2)-1} 
\pgfmathtruncatemacro{\cc}{random(1,2)} 
\pgfmathtruncatemacro{\v}{-random(1,4)+3}
%%%%% To define a function $f(x)=(x-a)^2-v$

% \pgfmathtruncatemacro{\absv}{abs(\v)}



\pgfmathtruncatemacro{\m}{random(2,1)-1}


\pgfmathtruncatemacro{\x}{\a-\m}
\pgfmathtruncatemacro{\y}{-\m^2+\v}


\pgfmathtruncatemacro{\s}{-random(2,4)}
\pgfmathtruncatemacro{\t}{random(1,5)-1}


\pgfmathtruncatemacro{\dotshape}{random(2,1)-1}

\pgfmathtruncatemacro{\ansy}{-(\bb-\a)^2+\v}
\pgfmathtruncatemacro{\ansx}{-(\bb-\a)^2+\v}

Use the graph of a function $f$ on the right to answer the following questions. 

\vspace{-\baselineskip}

\begin{minipage}[t]{\textwidth}
\begin{minipage}[c]{0.6\textwidth}
\begin{enumerate}[label={\textup(\arabic*)},afterlabel=~~~]
\item Find the value of $f(\bb)$.
\item Find the value of $x$ such that $f(x)=\t$.
\item Find the domain of the graph (write the domain in interval notation).
\item Find the range of the graph (write the range in interval notation).
\end{enumerate}
\end{minipage}\quad\quad
\begin{minipage}[c]{0.35\textwidth}
\vspace{2ex}
\begin{tikzpicture}[scale=0.75]
\begin{axis}[
 grid=both, 
 ymin=-5,ymax=5,
 xmax=5,xmin=-5,
 % xtick={-5,-4,...,5},
 % ytick={-5,-4,...,5},
 minor tick num=1,
]
\addplot[thick, restrict x to domain=\x:4.5, restrict y to domain=-5:4.5, name path=A, -stealth]   {-(x-\a)^2+\v};           
\addplot[thick, draw] ({\s-0.01},\t)--(\x, \y);
% \node[draw,shape=circle, minimum size=0.25mm,inner sep=0pt,outer sep=0pt,fill=black] at (\x,\y) {};
\ifnum\dotshape=0
\node[draw,shape=circle, minimum size=1.25mm,inner sep=0pt,outer sep=0pt] at (\s,\t) {}; 
\else
\node[draw,shape=circle, minimum size=1.25mm,inner sep=0pt,outer sep=0pt,fill=black] at (\s,\t) {};
\fi             
\end{axis}
\end{tikzpicture}
\end{minipage}
\end{minipage}


\begin{solution}\mbox{}
\begin{enumerate}[label={\textup(\arabic*)},afterlabel=~~~]
\item Since when the point $(\bb, \ansy)$ is on the graph of $f$, we know that \[f(\bb)= \ansy.\]
\item Since only $(\s,\t)$ is on the graph, equivalently, $f(\s)=\t$, we know that $x=\s$.
\item The domain is $\ifnum\dotshape=0(\else [\fi \s, +\infty)$.
\item The range is $(-\infty, \ifnum\t>\v {\t}\ifnum\dotshape=0 )\else ]\fi  \else \v ] \fi$.
\end{enumerate}
\end{solution}